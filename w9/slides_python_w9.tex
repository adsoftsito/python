% Author: Adolfo Centeno 
% Waves Lab

 
\documentclass{beamer}
\setbeamertemplate{navigation symbols}{}
\usepackage[utf8]{inputenc}
\usepackage{beamerthemeshadow}
\usepackage{listings}
\usepackage{hyperref}

\hypersetup{
  colorlinks=true,
  linkcolor=blue!50!red,
  urlcolor=green!70!black
}

\begin{document}
\title{Universidad Veracruzana}  
\subtitle{Centro de Investigaciones Cerebrales\\COMPUTACION CIENTIFICA Y BIOINFORMATICA}
\author{Adolfo Centeno, Gonzalo Aranda}
\date{\today} 

\begin{frame}
\titlepage
\end{frame}

\begin{frame}\frametitle{W9 - BioPython }
\tableofcontents
\end{frame} 


\section{Biopython (90m) }

\begin{frame}

Introduction to BioPython

\begin{enumerate}

\item

	Probar instalacion de biopython \href{https://github.com/adsoftsito/python/blob/master/w9/seq1.py}{seq1} 
	
\item
	Realizar ejercicios adicionales en BioPython
	
\end{enumerate} 


\end{frame}


\section{Receso  (20m) }

\begin{frame}


Receso ...

\end{frame}


\section{clustalw, muscle }

\begin{frame}

\begin{enumerate}

\item
   Revisar pagina unam con recursos en \href{https://www.ccg.unam.mx/~vinuesa/tlem/programa_TLEM.html}{bioinformatica}
\item 
    Probar instalacion de \href{http://www.clustal.org/}{clustal} (clustalo y clustalw)
\item 
	Probar instalacion de \href{https://www.drive5.com/muscle/}{muscle}
\item
	Push your work to \href{https://github.com}{github.com}

\end{enumerate} 


\end{frame}


\end{document}
