% Author: Adolfo Centeno 
% Waves Lab

 
\documentclass{beamer}
\setbeamertemplate{navigation symbols}{}
\usepackage[utf8]{inputenc}
\usepackage{beamerthemeshadow}
\usepackage{listings}
\usepackage{hyperref}

\hypersetup{
  colorlinks=true,
  linkcolor=blue!50!red,
  urlcolor=green!70!black
}

\begin{document}
\title{ITESM}  
\subtitle{Campus Puebla\\METODOS NUMERICOS EN INGENIERIA
}
\author{Adolfo Centeno}
\date{\today} 


\begin{frame}
\titlepage
\end{frame}

\begin{frame}\frametitle{Metodo de Gauss-Jordan}
\tableofcontents
\end{frame} 


\section{W8 Sesion 2 - Actividades Gauss-Jordan }

\begin{frame}

\textbf{W8 Sesion 2 - Actividades:}

\begin{enumerate}
\item
	Gauss-Jordan 2x2: \href{https://www.youtube.com/watch?v=10dlmC1MDco}{Video}.	
\item
	Gauss-Jordan 3x3: \href{https://www.youtube.com/watch?v=91xUg1L7O7s}{Video}.	

\item
	\href{https://github.com/adsoftsito/metodos-numericos/blob/master/w8/gaussjordan/gaussjor.pdf}{Codigo gauss-jordan} 

\item
    Tarea de investigacion parcial 2, de curso (\href{https://matlabacademy.mathworks.com/es}{matlab academy}) realizar : \\ 
    - Proyecto: Consumo eléctrico \\
	- Proyecto: Frecuencia de audio del curso 
	
\end{enumerate} 

\end{frame}


\section{W8 Sesion 2 - Tareas de Gauss-Jordan}

\begin{frame}


\textbf{W8 Sesion 2 - Tareas de Gauss-Jordan}


\begin{enumerate}
\item
	Gauss-Jordan 2x2: \href{https://www.youtube.com/watch?v=zn9BTnoGux8}{ejercicio 1}.	
\item
	Gauss-Jordan 3x3: \href{https://www.youtube.com/watch?v=em0ZWErimyU}{ejercicio 2}.	
\item
	Gauss-Jordan 4x4: \href{https://www.youtube.com/watch?v=uL3JwFy9BWA}{ejercicio 3}.	

\end{enumerate} 


\end{frame}




\end{document}
