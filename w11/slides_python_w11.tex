% Author: Adolfo Centeno 
% Waves Lab

 
\documentclass{beamer}
\setbeamertemplate{navigation symbols}{}
\usepackage[utf8]{inputenc}
\usepackage{beamerthemeshadow}
\usepackage{listings}
\usepackage{hyperref}

\hypersetup{
  colorlinks=true,
  linkcolor=blue!50!red,
  urlcolor=green!70!black
}

\begin{document}

\title{Universidad Veracruzana}  
\subtitle{Centro de Investigaciones Cerebrales\\COMPUTACION CIENTIFICA Y BIOINFORMATICA}
\author{Adolfo Centeno, Gonzalo Aranda}
\date{\today} 


\begin{frame}
\titlepage
\end{frame}

\begin{frame}\frametitle{Table of contents}
\tableofcontents
\end{frame} 


\section{W8 - Sesion 1 }

\begin{frame}

\textbf{W11 - Actividades:}

\begin{enumerate}

\item
	Revisar : \href{https://www.youtube.com/watch?v=XRcx8-2lLJI}{rubrica proyecto final}.	

\item Instalar  motor de Latex \href{https://miktex.org/download} {Windows}, \href {http://www.tug.org/mactex/mactex-download.html} {Mac}

\item Instalar editor de Latex \href{https://www.xm1math.net/texmaker/download.html}{texmaker}


\item
    Latex templates:  \\
    - \href{https://github.com/adsoftsito/python/blob/master/w11/simplepaper.tex}{Paper}. \\
    - \hrefhttps://github.com/adsoftsito/python/blob/master/w11/slides_python_w11.tex}{Slides} 

	

\end{enumerate} 

\end{frame}


\section{Break  (20m) }

\begin{frame}


Break ...

\end{frame}




\section{Autodock }

\begin{frame}

\begin{enumerate}

\item  Realizar Autodock \href{}{tutorial}   
\end{enumerate} 


\end{frame}



\end{document}
