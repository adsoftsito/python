% Author: Adolfo Centeno 
% Waves Lab

 
\documentclass{beamer}
\setbeamertemplate{navigation symbols}{}
\usepackage[utf8]{inputenc}
\usepackage{beamerthemeshadow}
\usepackage{listings}
\usepackage{hyperref}

\hypersetup{
  colorlinks=true,
  linkcolor=blue!50!red,
  urlcolor=green!70!black
}

\begin{document}
\title{Universidad Veracruzana}  
\subtitle{Centro de Investigaciones Cerebrales\\COMPUTACION CIENTIFICA Y BIOINFORMATICA}
\author{Adolfo Centeno, Gonzalo Aranda}
\date{\today} 

\begin{frame}
\titlepage
\end{frame}

\begin{frame}\frametitle{Table of contents}
\tableofcontents
\end{frame} 


\section{Presentacion (30m) }

\begin{frame} 

\begin{enumerate}
\item
 Presentacion de Profesor
\item
 Presentacion de Alumnos
\end{enumerate} 


\end{frame}


\section{Acceso Materiales del Curso (30) }

\begin{frame}

Materiales del Curso

\begin{enumerate}
\item
	Materiales disponbiles en \href{https://github.com/adsoftsito/python}{https://github.com/adsoftsito/python}.
\item
    Instalar git desde \href{https://git-scm.com/downloads}{https://git-scm.com/downloads}.
\item
	Clonar repositorio desde consola: \\ 
 	\textbf{git clone https://github.com/adsoftsito/python}
	

\end{enumerate} 

\end{frame}


\section{Acceso a Servidor (40m) }

\begin{frame}


Acceso a servidor por SSH

\begin{enumerate}
\item
	\textbf{cd python}
\item
	\textbf{cd ssh}
\item
	\textbf{ssh -i user user@104.198.244.0}
\end{enumerate} 


\end{frame}


\end{document}
